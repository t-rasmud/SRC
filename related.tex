\section{Background and Related Work\label{sec:related}}
Other researchers have also recognized the importance of the problem of nondeterminism.
Previous work in program analysis for nondeterminism has focused on unsound dynamic
approaches that identify flaky test cases.
NonDex~\cite{ShiGLM2016} uses a modified JVM that returns different results on different
executions, for a few key JDK methods with loose specifications.  Running a
test suite multiple times may reveal unwarranted dependence on those
methods.
%NonDex~\cite{ShiGLM2016} is a tool that manually\todo{This is confusing:  the
%  tool can't manually identify.  Do you mean that the developers manually
%  identified, then the tool uses those models?}
%identifies 
DeFlaker~\cite{BellLHEYM2018} looks at a range of commit versions
of a code, and marks a test as flaky if it doesn't execute any modified code but still fails in the newer version. These techniques
have been able to identify issues in real-world programs, some of which
have been fixed by the developers. Identifying and
resolving nondeterminism
earlier in the software development lifecycle is beneficial to
developers, because they can avoid bugs associated with flaky tests---reducing
costs~\cite{briski2008minimizing}.