\section{Evaluation\label{sec:results}}
To evaluate the usability of \theDeterminismChecker,
we applied it to the Randoop test
generator~\cite{PachecoLEB2007}.
Randoop is intended to be deterministic, when invoked on a deterministic
program~\cite{randoop-manual}.\footnote{Users of Randoop can pass in a different seed in order to
    obtain a different deterministic output.\todo{Could cut this footnote
      for space.}
%    Randoop has command-line
%    options that enable concurrency and timeouts, both of which can lead to
%    nondeterministic behavior.
}
% However, Randoop was not deterministic.  This caused the developers
% problems in 
% reproducing bugs reported by users, 
% reproducing test failures during development, and
% understanding the effect of changes to Randoop by comparing executions of two
% similar variants of Randoop.
The developers had recently
spent two weeks of full-time work to
eliminate unintentional nondeterministic behavior, by repeatedly
running Randoop with verbose logging enabled and
looking for differences in logging output (personal communication, 2019).

First, we wrote specifications for libraries Randoop uses, such as the JDK,
JUnit, and plume-util.

Second, we wrote type qualifiers in the Randoop source code to express its
determinism specification.

Third, we ran
\theDeterminismChecker.  Each warning indicated a mismatch between the
specification and the implementation.  We addressed each warning by changing our
specification, reporting a bug in Randoop, or suppressing a false positive warning.
At the end of the process, Randoop's 24K lines of code contained
98 \<@NonDet>,
64 \<@OrderNonDet>,
1083 \<@Det>,
and 103 \<@PolyDet>\todo{The paper doesn't mention polymorphism.  Is the
  breakdown of the different types of annotations interesting to readers?} annotations.
% The other types were given default type qualifiers by \theDeterminismChecker.
This is about 1 annotation per 17 lines of code,
% Justification for the claim about generics:
% There are about 1389 lines with uses of generics, according to the following:
% (rg "<.*>" "everything" (substitute-in-file-name "$test/randoop/src/main/java/"))
% (delete-matching-lines ": /?*\\(@Option\\|//\\|/?\\*\\|return \"\\|\"\\)" nil nil t)
% (replace-regexp " [<>] " "")
% (replace-regexp "[<>]=" "")
% (replace-regexp "[-=][<>]" "")
% (replace-sring "<init>" "")
% (delete-non-matching-lines "[<>]" nil nil t)
and is fewer than the uses
of Java generic types in Randoop.

\TheDeterminismChecker found \numRandoopBugs previously-unknown
nondeterminism bugs in Randoop.
The developers accepted our bug reports and committed fixes to the repository. 
Examples
of severe bugs follow:

\begin{itemize}
    \item
    \textbf{HashSet bug}: In Randoop's input (called the ``code under test''),
    if a type variable's lower or upper
    bound has a type parameter that the type variable itself does not have,
    then Randoop is nondeterministic.
    This situation does occur, even in Randoop's test suite.
    The developers fixed this by changing a \<HashSet> to \<LinkedHashSet>
    (commit c975a9f7, shown in \cref{fig:randoop-bug-hashset}\todo{Remove
      the figure and instead give examples of specified code.  Or, show
      just the ``before'' version of the code \emph{with annotations},
      which will demonstrate how Randoop issues warnings.}).
    \TheDeterminismChecker could detect this bug by assigning\todo{A
      type-checker does not assign a type} the type qualifier \<OrderNonDet>
    to the newly created \<HashSet> (\<parameters>)
    \todo{This sentence is impossible to follow.  Each of
      the two sentences
      vaguely describes some difference between two versions of code.
      Don't do that.  (A writing rule is ``show, don't tell''.)  Instead,
      show the actual annotated code, and explain the warning that
      \theDeterminismChecker issues.}
    on line 161. The function returns an \<ArrayList> which gets the
    determinism type \<Det>.
    \todo{The following sentence is also unclear.  Who is the actor?  What
      is the code?  Show it rather than giving this hard-to-follow explanation.}Creating a \<Det ArrayList> with 
    \<OrderNonDet> arguments violates the typing rules causing \theDeterminismChecker to report a bug.
    
%    \TheDeterminismChecker confirmed that 
%    25 other uses of \<new HashSet> were acceptable, as were 15 uses of \<new HashMap>.
    \item
    \textbf{Classpath bug}
    Randoop used the CLASSPATH environment variable in preference to the
    classpath passed on the command line.
    This can cause incorrect behavior, both in Randoop's test suite and in the field,
    if a user sets the environment variable.
    The developers fixed the problem by changing Randoop to not read the environment variable
    (commit 330e3c56).
%    \TheDeterminismChecker verified that all other uses of system and Java
%    properties did not lead to nondeterministic behavior.
\end{itemize}




\section{Research Contributions\label{sec:contributions}}
%This paper makes the following contributions:
\begin{enumerate}
  \item We designed a type system for expressing determinism properties.

  \item We implemented the analysis, as a pluggable type system for Java in a
    tool called \theDeterminismChecker.

  \item In a case study, we ran our analysis on a 24 KLOC project.
%  that
%    developers had already spent weeks of testing and inspection effort to
%    make deterministic.  
    \TheDeterminismChecker
    found 5 instances of nondeterminism that the developers had
    overlooked.
\end{enumerate}


%%  LocalWords:  util NonDet OrderNonDet Det PolyDet c975a9f7 e3c56
