\section{Evaluation\label{sec:results}}
To evaluate the usability of \theDeterminismChecker,
we applied it to the Randoop test
generator~\ifanonymous\citepalias{randoop-tool}\else\cite{PachecoLEB2007}\fi.
We also wrote specifications for libraries Randoop uses, such as the JDK,
JUnit, plume-util, and others.

Randoop is intended to be deterministic, when invoked on a deterministic
program~\cite{randoop-manual}.\footnote{Users of Randoop can pass in a different seed in order to
    obtain a different deterministic output.  Randoop has command-line
    options that enable concurrency and timeouts, both of which can lead to
    nondeterministic behavior.}
However, Randoop was not deterministic.  This caused the developers
problems in 
reproducing bugs reported by users, 
reproducing test failures during development, and
understanding the effect of changes to Randoop by comparing executions of two
similar variants of Randoop.

In brief, we wrote type qualifiers in the Randoop source code to express its
determinism specification,
then ran
\theDeterminismChecker.  Each warning indicated a mismatch between the
specification and the implementation.  We addressed each warning by changing our
specification, reporting a bug in Randoop, or suppressing a false positive warning.

TheDeterminismChecker found \numRandoopBugs previously-unknown nondeterminism bugs in Randoop.
The Randoop developers accepted our bug reports and committed fixes to the repository. An example
of a severe bugs follows, according to the Randoop developers' categorization:

\begin{itemize}
    \item
    \textbf{HashSet bug}: One use of \<HashSet> could cause a problem if a type variable's lower or upper
    bound in the code Randoop is run on has a type parameter that the type variable itself does not have.
    This situation does occur, even in Randoop's test suite.
    The developers fixed this by changing a \<HashSet> to \<LinkedHashSet>
    (commit c975a9f7, shown in \cref{fig:randoop-bug-hashset}).
    \TheDeterminismChecker confirmed that 
    25 other uses of \<new HashSet> were acceptable, as were 15 uses of \<new HashMap>.
\end{itemize}

\section{Research Contributions\label{sec:contributions}}
This paper makes the following contributions:
\begin{enumerate}
  \item We designed a type system for expressing determinism properties.

  \item We implemented the analysis, as a pluggable type system for Java, in a
    tool called \theDeterminismChecker.

  \item In a case study, we ran our analysis on a 24 KLOC project that
    developers had already spent weeks of testing and inspection effort to
    make deterministic.  \TheDeterminismChecker
    discovered 5 instances of nondeterminism that the developers had
    overlooked.