\section{Introduction\label{sec:introduction}}

A nondeterministic program may produce different output on different runs
when provided with the same input and is a serious problem for software developers and users.
\begin{itemize}
\item
  Nondeterminism makes a program difficult to \textbf{test}, because test
  oracles must account for all possible behaviors while still enforcing
  correct behaviors~\cite{LuoHEM2014,ShiGLM2016,BellLHEYM2018,Sudarshan}.
\item
  Nondeterminism makes it difficult to \textbf{compare} two runs of a
  program on different data, or to compare a run of a slightly modified
  program to an original program.  This hinders debugging and maintenance,
  and prevents use of techniques such as Delta Debugging~\cite{Zeller1999,YuLCZ2012}.
\end{itemize}

Two well-known sources of nondeterminism are concurrency
and coin-flipping
(calls to a \<random> API\@).
It may be surprising that nondeterminism is common even in sequential
programs that do not flip coins.
For example, a program that iterates over a hash table
may produce different output on different runs.
So may any program that uses default formatting, such as Java's
\<Object.toString()>, which includes a memory address.
Other nondeterministic APIs include date-and-time functions and
accessing system properties such as the file system or environment variables.

We have created an analysis that detects nondeterminism or verifies its
absence in sequential programs.
Our analysis permits a programmer to specify which parts of their program
are intentionally nondeterministic, and it verifies that the remainder is deterministic.
Our analysis works at compile time, giving a guarantee over every possible
execution of the program, unlike unsound dynamic tools that attempt
to discover when a program has exhibited nondeterministic behavior on a
specific run.  
Our analysis handles collections that will contain the same values, but
possibly in a different order, on different runs.
Our analysis permits calls to
nondeterministic APIs, and only issues a warning if they are used in ways
that may lead to nondeterministic output observed by a user.  Like any
sound analysis, it can issue false positive warnings.

The high-level goal of our work is to provide programmers with a tool for
specifying deterministic properties in a program and verifying them
statically.

\begin{figure*}

\noindent
In \<TypeVariable.java>:

\begin{Verbatim}
160:   public List<TypeVariable> getTypeParameters() {
161:-    Set<TypeVariable> parameters = new HashSet<>(super.getTypeParameters());
161:+    Set<TypeVariable> parameters = new LinkedHashSet<>(super.getTypeParameters());
162:     parameters.add(this);
163:     return new ArrayList<>(parameters);
164:   }
\end{Verbatim}

\caption{Fixes made by the Randoop developers in response to our bug report
  about improper use of a HashSet.  Lines starting with ``\<->'' were
 removed and those starting with ``\<+>'' were added.
 Our tool, \theDeterminismChecker, confirmed that 
25 other uses of \<new HashSet> were acceptable, as were 15 uses of \<new HashMap>.}
\label{fig:randoop-bug-hashset}
\end{figure*}

%\todo{It is essential that the introduction includes an example real-world
%  defect that \theDeterminismChecker found.}


% LocalWords:  NonDex DeFlaker Det OrderNonDet NonDet
