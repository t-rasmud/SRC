\section{Introduction\label{sec:introduction}}

A nondeterministic program may produce different output on different runs
when provided with the same input.
For example, a program that iterates over a hash table
may produce different output on different runs.
%So may any program that uses default formatting, such as 
Other sources of nondeterministic include
% default formatting (Java's \<Object.toString()> includes a memory address),
date-and-time functions,
accessing system properties,
concurrency,
and \<random>.
%
Nondeterminism is a serious problem for software developers and users.
\begin{itemize}
\item
  Nondeterminism makes a program difficult to \textbf{test}, because test
  oracles must account for all possible behaviors while still enforcing
  correct behaviors~\cite{LuoHEM2014,ShiGLM2016,BellLHEYM2018,Sudarshan}.
\item
  Nondeterminism makes it difficult to \textbf{compare} two runs of a
  program on different data, or to compare a run of a slightly modified
  program to an original program.  This hinders debugging and maintenance,
  and prevents use of techniques like Delta Debugging~\cite{Zeller1999,YuLCZ2012}.
\end{itemize}

% such as the file system or environment variables.

We have created a specification language that lets programmers specify
determinism properties.
Some parts of the program may be intentionally nondeterministic, so long
as they do not lead to nondeterministic output observed by a user.  

We have created a sound static analysis that verifies the program against the specification.
If our analysis issues no warnings, then the program produces the same
output when executed twice on the same inputs.  This guarantee is modulo
the limitations of the
analysis, notably \todo{Remove this limitation?}concurrency, implicit control flow, and unanalyzed libraries.
Like any
sound analysis, it can issue false positive warnings.

\begin{figure*}

\noindent
In \<TypeVariable.java>:

\begin{Verbatim}
160:   public List<TypeVariable> getTypeParameters() {
161:-    Set<TypeVariable> parameters = new HashSet<>(super.getTypeParameters());
161:+    Set<TypeVariable> parameters = new LinkedHashSet<>(super.getTypeParameters());
162:     parameters.add(this);
163:     return new ArrayList<>(parameters);
164:   }
\end{Verbatim}

\caption{Fixes made by the Randoop developers in response to our bug report
  about improper use of a HashSet.  Lines starting with ``\<->'' were
 removed and those starting with ``\<+>'' were added.}
\label{fig:randoop-bug-hashset}
\end{figure*}


% LocalWords:  NonDex DeFlaker Det OrderNonDet NonDet unanalyzed
%%  LocalWords:  TypeVariable
